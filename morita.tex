\documentclass[shinchoku]{hitsotsuron} %文書のクラスファイルを指定

%題目の各文字列を指定
\title{セキュアコーディング支援システムの開発}

\papertype{平成28年度卒業研究概要}
\author{森田浩平}
\snumber{B214205}
\advisor{松本慎平}
\date{2017年9月21日}

\begin{document} %文書本体の開始

%題目を実際に作成
\twocolumn[%
\maketitle
]

\section{諸言}

これはMarkdownでTexを生成し,PDFへとエクスポートするサンプルです.\\
このサイト\cite{siteA}やこのリポジトリ\cite{siteB}を参考にしています.


\section{提案}

テキスト


\subsection{サービスの概要}
テキスト


\subsection{アーキテクチャ}
テキスト


\section{調査}
テキスト


\section{結言}
テキスト.


\begin{thebibliography}{1}

\bibitem{siteA}
markdown->TeXの変換だけで卒論を仕上げるのに便利だったツール10個まとめ,http://mizchi.hatenablog.com/entry/2014/01/20/090957

\bibitem{siteB}
orumin/thesis\_template,https://github.com/orumin/thesis\_template

\end{thebibliography}

\end{document} 
